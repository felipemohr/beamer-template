%*----------- SLIDE -------------------------------------------------------------
\begin{frame}[t]{Finalização}
    \begin{itemize}
        \item Cada líder deverá realizar a apresentação final do desafio no dia 25/maio/2020.
        \item No dia da apresentação, somente o líder poderá responder os questionamentos emitidos pelos facilitadores.
        \item A avaliação será da equipe, não havendo avaliação individual dos integrantes da equipe com exceção do líder de cada equipe.
        \item A apresentação deverá ser desenvolvida em latex.
        \item Os videos dos desafios deverão estar contidos na apresentação final.
        \item Os videos deverão ser completos, tendo começo, meio e fim da missão realizada.
    \end{itemize}
%*----------- notes
    \note[item]{Notes can help you to remember important information. Turn on the notes option.}
\end{frame}
%-
%*----------- SLIDE -------------------------------------------------------------
\begin{frame}[c]{A importância atual da robótica}
    \begin{center}
        % \movie[loop,width=0.6\linewidth,height=0.3375\linewidth,showcontrols=false,autostart]{\includegraphics[width=0.6\textwidth]{Media/gifs/robotdesinfec0.png}}{Media/gifs/robotdesinfec.wmv}
    
        \includemedia[
            width=0.7\linewidth,
            totalheight=0.39375\linewidth,
            activate=pageopen,
            passcontext, 
            %transparent,
            addresource=./Media/gifs/robotdesinfec.wmv,
            flashvars={
            source=./Media/gifs/robotdesinfec.wmv
            &autoPlay=true
            &autoRewind=true
            &loop=true}
            ]{\fbox{\includegraphics{Media/gifs/robotdesinfec0.png}}}{VPlayer.swf}
    \end{center}
%  %\movie[width=8cm,height=4.5cm]{test}{../Movies/Darwin-OP.mp4}

% % \includemedia[
% %     width=0.4\linewidth,
% %     totalheight=0.225\linewidth,
% %     activate=pageopen,
% %     passcontext,  %show VPlayer's right-click menu
% %     addresource=../Movies/Darwin-OP.mp4,
% %     flashvars={
% %       %important: same path as in `addresource'
% %       source=../Movies/Darwin-OP.mp4
% %     }
% %   ]{\fbox{Click!}}{VPlayer.swf}

%     \pdfpcmovie{\includegraphics[width=\textwidth]{Darwin-OP}}{Darwin-OP.mp4}

%*----------- notes
    \note[item]{Notes can help you to remember important information. Turn on the notes option.}
 \end{frame}
%-
%*----------- SLIDE -------------------------------------------------------------
\begin{frame}[fragile]{A importância atual da robótica}
    Para a implentação de R gráficos deve-se realizar os seguintes comando no ambiente R:
    \begin{lstlisting}[language=R]
        library(tikzDevice)
        beamer.parms = list(paperwidth   = 364.19536/72,
                    paperheight  = 273.14662/72,
                    textwidth    = 307.28987/72,
                    textheight   = 269.14662/72)
        tikz("./your_file.tex", 
            width = beamer.parms$textwidth, 
            height = beamer.parms$textheight)
        ggqqplot(na.omit(my_data$col2))
        dev.off()
    \end{lstlisting}

    \begin{columns}
        \column{.01\textwidth}
        \column{.59\textwidth}
            \textbf{A penúltima linha do texto acima é o códifo em R para a construção do gráfico.}
                
        \column{.4\textwidth}
            \centering
            \begin{tikzpicture}[thick, scale=0.4, every node/.style={scale=0.1}]
                \node[at=(current page.center)] {
               %\input{./Media/r-graphics/img-marco1.tex}
                \input{./Media/r-graphics/graficox.tex}
                };
            \end{tikzpicture}
    \end{columns}
%*----------- notes
    \note[item]{Notes can help you to remember important information. Turn on the notes option.}
 \end{frame}
%-
%*----------- SLIDE -------------------------------------------------------------
\begin{frame}[c]{A importância atual da robótica}
    \centering
    \begin{bclogo}[ 
        couleur=white!10!white,
        ombre=false,
        epBord=3,
        couleurBord = gcolor,
        arrondi = 0.2,
        logo=\bcinfo]{}
        \centering
        \begin{tikzpicture}[thick, scale=0.35, every node/.style={scale=0.5}]
            \node[at=(current page.center)] {
            \input{./Media/r-graphics/graficox.tex}
            };
        \end{tikzpicture}
    \end{bclogo}
%*----------- notes
    \note[item]{Notes can help you to remember important information. Turn on the notes option.}
 \end{frame}
 %-
 %*----------- SLIDE -------------------------------------------------------------
\begin{frame}
    \begin{center}
        \vspace*{1.5cm}
        \textbf{\Huge{\textcolor{lcolor}{MUDANÇA}}}
    \end{center}
    
%*----------- notes
    \note[item]{Notes can help you to remember important information. Turn on the notes option.}
 \end{frame}
 %-
 %*----------- SLIDE -------------------------------------------------------------
\begin{frame}

    %\hspace*{-1cm}
    \begin{columns}
        %\column{.01\textwidth}
        \column{0.4\textwidth}
        ~\hfill
            \begin{beamercolorbox}[sep=8em, colsep*=18pt, center, wd=\textwidth,ht=\paperheight]{title page header}
                \begin{center}
                    \textbf{\huge{VISÃO}}\par
                    \vspace*{0.3cm}
                    \textbf{\huge{FUTURA}}
                \end{center}
                 
            \end{beamercolorbox}%
        \column{.05\textwidth} 
        \column{.6\textwidth}
            \begin{itemize}
                \item marco1
                \item marco 
                \item kajsflksadjlkf
                \item i
            \end{itemize}
    \end{columns}
  
%*----------- notes
    \note[item]{Notes can help you to remember important information. Turn on the notes option.}
 \end{frame}
 %-